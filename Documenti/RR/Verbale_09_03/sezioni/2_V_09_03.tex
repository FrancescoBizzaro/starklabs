\section{Questioni emerse}
Di seguito sono trascritti gli argomenti che il gruppo ha trattato durante la 
riunione con la Proponente Mivoq. Il meeting ha avuto carattere informale. \\ A 
seguire, vengono riportate le domande poste accompagnate da relativa 
formalizzazione della risposta del referente dott. Giulio Paci.

\subsection{Quale obiettivo si vuole raggiungere con l’applicazione?}
Il capitolato è rimasto volutamente vago, per non imporre alcun limite sul tipo 
di applicazione. L'unico requisito primario è l'utilizzo del motore di sintesi, 
con lo scopo ultimo di metterne in risalto le potenzialità. Tuttavia è 
fortemente desiderabile che l'applicazione si dimostri utile per qualche 
utilizzo reale, in linea con gli esempi suggeriti nel capitolato.


\subsection{Per quale motivo si è scelto il mondo mobile?}
La scelta è stata dettata dal desiderio del committente di entrare in tale 
settore. La scelta è stata motivata, durante l'incontro, con le seguenti 
motivazioni:
\begin{itemize}
\item incentivare la diffusione dell'applicazione;
\item praticità nell'avere l'applicativo su un dispositivo mobile, il quale 
risulta facilmente trasportabile per eventuali dimostrazioni della tecnologia 
FA-TTS\G.
\end{itemize}


\subsection{Si richiede che l'applicazione elabori una frase in tempo reale, 
senza usare audio creato precedentemente?}
Possono essere fatte entrambe le cose, purché l'audio venga creato attraverso 
il motore di sintesi FA-TTS\G\ con una connessione HTTP.


\subsection{Supposto di utilizzare il motore di sintesi in un’applicazione di messaggistica, quali sarebbero i vantaggi offerti rispetto all’invio di note vocali?}
Una nota vocale può essere solamente ascoltata, mentre attraverso la sintesi 
vocale un utente sarebbe libero di scegliere se leggere il messaggio o 
ascoltarlo sintetizzato, magari con la voce digitale del mittente. Di solito si 
preferisce leggere i messaggi, ma in alcune situazioni  questo non è possibile 
(ad esempio se si sta guidando). A questo punto risulta utile avere la 
possibilità di farsi leggere il messaggio da un servizio TTS\G.

\subsection{Quali sono le principali problematiche riguardanti la connessione al servizio remoto?}
Il servizio è offerto tramite un server interrogabile attraverso un'interfaccia 
HTTP\G. Sono previste chiamate HTTP\G\ in grado di fornire l'elenco delle voci 
disponibili, con le varie impostazioni modificabili. \\
Le problematiche possono essere di vario genere:
\begin{itemize}
\item la connessione può cadere;
\item i tempi di risposta della connessione potrebbero non essere ottimali:
\begin{itemize}
\item[-]i file da trasferire sono codificati in formato WAV\G\ e pertanto 
bisogna tenere conto anche del loro peso;
\item[-]l'elaborazione richiede tempo.
\end{itemize}
\end{itemize}
Per poter arginare il problema del ritardo si potrebbero adottare strategie di 
\textit{caching}\G , conoscendo con anticipo i possibili testi da elaborare.


\subsection{Fino a che punto il modulo di sintesi deve integrarsi con il 
sistema? Quali \textit{features} devono essere sempre garantite 
dall'applicazione?}
\'E desiderabile che venga realizzato un modulo software capace di 
integrare il servizio remoto offerto con il sistema mobile nel quale viene 
eseguita l'applicazione. Per risolvere la possibile assenza di connessione, che 
renderebbe impossibile lo sfruttamento del motore di sintesi, devono essere 
implementati meccanismi di \textit{fallback}\G\ che utilizzino le voci presenti 
nel sistema. In questo modo sarà possibile effettuare la sintesi con le voci di 
sistema e consentire all'applicazione di funzionare correttamente.

\subsection{\'E possibile fare maggior chiarezza sulla richiesta di 
un'implementazione separata delle varie componenti dell'applicazione?}
L'applicazione deve essere formata da 4 parti:
\begin{itemize}
	\item un modulo per la sintesi, a sé stante, che corrisponde 
	all'implementazione del motore FA-TTS\G\ di Mivoq;
	\item un'applicazione per la configurazione, che deve essere in grado di 
	interagire con il modulo per la sintesi al fine di modificarne determinati 
	parametri. Ad esempio si vuole fornire la possibilità di aggiungere nuove 
	voci, assieme a nuovi \textit{preset}\G\ di effetti associabili alle voci 
	date;
	\item una libreria che faciliti l'utilizzo delle funzionalità aggiuntive, 
	permettendo il riuso della libreria stessa;
	\item un'applicazione innovativa che dimostri in modo chiaro le 
	potenzialità offerte dal suddetto modulo di sintesi.
\end{itemize}


\subsection{L'applicazione dedicata alla configurazione delle impostazioni del 
motore si può integrare nelle impostazioni di sistema di Android?}
\'E necessario che il gruppo approfondisca tale argomento autonomamente. 
Probabilmente non è possibile implementare tutte le funzionalità richieste 
direttamente nelle impostazioni del sistema operativo. Di conseguenza 
l'applicazione di configurazione dovrà svolgere questo compito esternamente. 
Tale applicazione dovrà consentire all'utente di aggiungere voci in maniera 
dinamica e di salvarle con i loro parametri all'interno del sistema. \'E quindi 
possibile sviluppare due applicazioni, oppure combinare le due funzionalità in 
un'unica applicazione.

\subsection{\'E possibile velocizzare il tempo di campionamento della voce 
abbassando la durata del processo a un massimo di 10 minuti?}
Attualmente il \textit{competitor}\G\ principale di Mivoq è circa 16 
volte più lento rispetto ai 45 minuti richiesti dal motore FA-TTS\G\ per 
registrare una nuova voce. Il motivo per cui il campionamento è così lungo 
deriva dalla necessità di raccogliere ed elaborare un elevato numero di dati, 
tant'è che solitamente occorrono più di 100 frasi per ottenere un buon 
risultato. Il sistema di registrazione offerto è ancora nelle fasi preliminari 
di sviluppo e al momento i tempi dichiarati sono i migliori disponibili.

\subsection{Si riesce ad ottenere un buon risultato registrando nuove voci dal 
microfono di uno \textit{smartphone}\G?}
La strumentazione degli \textit{smartphone}\G\ odierni è adatta ad ottenere un 
buon campionamento, il reale problema è dato dai tempi troppo elevati che 
interessano il processo.

\subsection{Quali strumenti di lavoro e di gestione ci verranno offerti?}
Mivoq fornisce il motore di sintesi assieme alle voci predefinite, materiale 
che può essere scaricato direttamente dal sito dell'azienda. Su richiesta del 
gruppo sarà possibile avere accesso ad un sito interno di Mivoq per campionare 
la propria voce e poterla riutilizzare anche all'interno dell'applicazione da 
realizzare. Non viene dato alcun vincolo sugli strumenti di sviluppo per 
implementare le applicazioni richieste. Sarà compito del team lo studio di 
eventuali \textit{framework}\G\ per facilitare il processo di sviluppo 
dell'applicazione finale.

